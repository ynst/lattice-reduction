
\documentclass[11pt]{article}
%-------Packages---------
\usepackage{amssymb,amsfonts,amsmath,amsthm}
\usepackage{geometry}
 \geometry{
 total={170mm, 220mm},
 left=20mm,
 top=30mm,
 bottom=40mm
 }
 \usepackage{float}
\usepackage[ampersand]{easylist}
\usepackage{color}
\usepackage{graphicx}
\usepackage[font = small]{caption}
\usepackage{subcaption}
\usepackage{dsfont}
%\usepackage[section]{placeins}
\usepackage{enumerate}
\usepackage{mathrsfs}
\usepackage{verbatim}
\usepackage{clrscode}
\usepackage{listings}
\lstset{ 
        language=Matlab,                                % choose the language of the code
%       basicstyle=10pt,                                % the size of the fonts that are used for the code
        numbers=left,                                   % where to put the line-numbers
        numberstyle=\footnotesize,                      % the size of the fonts that are used for the line-numbers
        stepnumber=1,                                           % the step between two line-numbers. If it's 1 each line will be numbered
        numbersep=5pt,                                  % how far the line-numbers are from the code
%       backgroundcolor=\color{white},          % choose the background color. You must add \usepackage{color}
        showspaces=false,                               % show spaces adding particular underscores
        showstringspaces=false,                         % underline spaces within strings
        showtabs=false,                                         % show tabs within strings adding particular underscores
%       frame=single,                                           % adds a frame around the code
%       tabsize=2,                                              % sets default tabsize to 2 spaces
%       captionpos=b,                                           % sets the caption-position to bottom
        breaklines=true,                                        % sets automatic line breaking
        breakatwhitespace=false,                        % sets if automatic breaks should only happen at whitespace
        escapeinside={\%*}{*)}                          % if you want to add a comment within your code
}


\usepackage{tikz}
\usetikzlibrary{arrows}
\usepackage{upgreek}
\usepackage{bm}
\usepackage[ruled]{algorithm2e}
\newtheorem{theorem}{Theorem}[section]
\newtheorem{lemma}[theorem]{Lemma}
\newtheorem{definition}[theorem]{Definition}
\newtheorem{note}[theorem]{Note}
\newtheorem{example}[theorem]{Example}
%--------New Commands-------------------
\linespread{1.05}

\DeclareMathOperator{\tr }{Tr}

\newcommand{\e}{\mathrm{e}}
\newcommand{\sub}{\subseteq}

\renewcommand\Re{\operatorname{Re}}
\renewcommand\Im{\operatorname{Im}}
\newcommand{\res}{\text{res}}
\newcommand{\Lagr}{\mathcal{L}}
\newcommand{\ceil}[1]{\lceil #1 \rceil}
\newcommand{\floor}[1]{\lfloor #1 \rfloor}
\newcommand{\ang}[1]{\langle #1 \rangle}
\newcommand{\set}[1]{\{ #1 \}}
\newcommand{\st}{\text{ s.t. }}
%\newcommand{\iff}{\text{ iff }}
\newcommand{\Times}{\cdot}
\newcommand{\pdd}[2]{\frac{\partial #1}{\partial #2}}
\newcommand{\dpdd}[2]{\frac{\partial^2 #1}{\partial #2 ^2}}
\newcommand{\mpd}[3]{\frac{\partial^2 #1}{\partial #2 \partial #3}}
\newcommand{\proj}[2]{\text{Proj}_{#1}(#2)}
\newcommand{\intpi}[1]{\int_{-\pi}^\pi #1 dx}
\newcommand{\intone}[1]{\int_{-1}^1 #1 dx}
\newcommand{\valpi}{|^{\pi}_{-\pi}}
\newcommand{\bolda}{\textbf{a}}
\newcommand{\Lra}{\Leftrightarrow}
\newcommand{\C}{\mathbb{C}}
\newcommand{\Z}{\mathbb{Z}}
\newcommand{\D}{\mathbb{D}}
\newcommand{\Q}{\mathbb{Q}}
\newcommand{\R}{\mathbb{R}}
\renewcommand{\H}{\mathbb{H}}
\renewcommand{\S}{\mathbb{S}}
\newcommand{\X}{\chi}
\newcommand{\crazyU}{\mathfrak{A}}
\newcommand{\dist}{\text{dist}}
\newcommand{\ol}{\overline}
\newcommand{\Aut}{\text{Aut}}
\newcommand{\vv}{\textbf{v}}
\newcommand{\xx}{\textbf{x}}
\newcommand{\hh}{\textbf{h}}
\newcommand{\EE}{\textbf{E}}
\newcommand{\Var}{\textbf{Var}}
\newcommand{\summ}{\sum_{i=1}^n}
\newcommand{\pdf}{probability density function}
\newcommand{\I}{\textbf{I}}


\newcommand{\Cov}{\textbf{Cov}}
\newcommand{\x}{x^*}
\newcommand{\argmin}{\operatornamewithlimits{argmin}}
\newcommand{\argmax}{\operatornamewithlimits{argmax}}
%\newcommand{\tr}{\text{Tr}}
\newcommand{\pageline}{\noindent\rule[0.5ex]{\linewidth}{0.5pt}}
\newcommand{\grad}{\nabla}
%\renewcommand{\varphi}{\varphi}
\newcommand{\liminfty}[1]{\lim_{#1 \rightarrow \infty}}
\title{Computational Experiments of AE}
\author{Yunus Tuncbilek}
%{\color[rgb]{0.000000,0.000000,0.000000}
\begin{document}
\maketitle
   
\section{Letchford's test instances}
\textbf{Number of facilities:} In his test instances, Letchford has $n$ clients and $m$ facilities and sets $m=n$. \textbf{Assignment (transaction) costs} Each facility and customer location is set to a random point on the unit square, therefore setting each assignment cost to the euclidian distance between their locations on the unit square.
\subsection{Description}
There are four different test cases depending on how large the facility costs are and whether they are randomized or constant. Small and constant facility costs, medium and constant, large and constant, and entirely varied. 

$m,n$ are the number of facilities and the right columns  are the time both algorithms take in seconds (according to what Letchford reports in his paper).

\subsection*{Small facility costs}
All facility (fixed) costs are set to $\sqrt{n}/1000$ 
Letchford:
\begin{center}
\begin{tabular}{|rlc|clclcl}
  \hline
  m=n & Reduction time 1 (s) & Eliminated & Reduction time 2 (s) & Eliminated \\ \hline
  500 & 0.01 & 97 & 0.03 & 97\\ \hline
  1000 & 0.03 & 99 & 0.52 & 99\\ \hline
  2000  & 0.23 & 98& 2.56 &98 \\ \hline
\end{tabular}
\end{center}
AE:
\begin{center}
\begin{tabular}{|rlc|clclcl}
  \hline
  m=n & Time (s) & \%Eliminated \\ \hline
  500 & 0.128 & 29  \\ \hline
  1000 & 0.445 & 12 \\ \hline
  2000  & 1.016 & 0.6 \\ \hline
\end{tabular}
\end{center}

\begin {lemma}
The reason why AE's reduction rates are so low is the following: since SPLP is submodular, in the first step of the reduction procedure, if \[\Pi(0, \cdots, 1, \cdots, 0) - \Pi(0, \cdots, 0)\] is less than zero, we could set the entry of 1 to 0 permanently. However, since in SPLP there is no capacity on how many clients a facility can serve, opening a new facility while all of the others are closed leads to a situation where every client is served by this new facility. Considering the assignment profits are random numbers between 0 and 1, the total gain from opening the new facility is on average greater than $\frac{n}{2}$, which becomes greater than $\sqrt{n}/1000$ as n gets large. So, AE's infimum test cannot reduce well in SPLP.
\end{lemma}

\subsection*{Medium facility costs}
All facility costs are set to $\sqrt{n}/100$\\
Letchford:
\begin{center}
\begin{tabular}{|rlc|clclcl}
  \hline
  m=n & Reduction 1 (s) & Eliminated & Reduction 2 (s) & Eliminated \\ \hline
  500 &0.02& 89& 0.04 & 96\\ \hline
  1000 & 0.42 & 76 & 0.67 & 96\\ \hline
  2000  & 0.92 & 71 & 1.45 & 97\\ \hline
\end{tabular}
\end{center}
AE:
\begin{center}
\begin{tabular}{|rlc|clclcl}
  \hline
  m=n & Time (s) & \%Eliminated \\ \hline
  500 &  0.097 &  2.400  \\ \hline
  1000 & 0.875 & 2.700 \\ \hline
  2000  & 1.767  & 0.450\\ \hline
\end{tabular}
\end{center}

\subsection*{Large facility costs}
All facility costs are set to $\sqrt{n}/10$ 
AE:
\begin{center}
\begin{tabular}{|r|l|l|}
  \hline
  m=n & Time (s) & Eliminated \\ \hline
  500 & 0.036 & 56.200  \\ \hline
  1000 & 0.156 & 22.300\\ \hline
  2000  & 0.787 & 1.9 \\ \hline
\end{tabular}
\end{center}
\section{Other test cases}
\subsection*{Quadratic cases}
In the cases where the fixed cost is $k\sqrt{n}$, where $k$ is a constant less than $0.0001$, AE can reduce due to the initial supremum check.\\
The following case: the fixed costs are all constant and equal to $\frac{\sqrt{n}}{100,000}$.
AE:
\begin{center}
\begin{tabular}{|r|l|l|l}
  \hline
  m=n & Time (s) & Eliminated \\ \hline
  1000 & 0.170s & 99\\ \hline
  5000 & 10 s & 92 \\ \hline
\end{tabular}
\end{center}

\subsection*{Linear-fixed cost cases}
When the fixed cost is equal to $\frac{n}{2}$, we get the following
AE:
\begin{center}
\begin{tabular}{|r|l|l|}
  \hline
  m=n & Time (s) & Eliminated \\ \hline
  4000 & 26 s & 76  \\ \hline
\end{tabular}
\end{center}
Fixed cost $= 0.96n$ gives even better results but is a very unrealistic case.
\begin{center}
\begin{tabular}{|r|l|l|}
  \hline
  m=n & Time (s) & Eliminated \\ \hline
  500 & 0.032 s & 87  \\ \hline
  1000 & 0.098 s & 91 \\ \hline
  2000 &  0.558 s & 79 \\ \hline
  4000 & 2.188 s & 79 \\ \hline
  \end{tabular}
\end{center}
\end{document}

